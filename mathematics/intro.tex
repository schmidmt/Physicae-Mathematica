\chapter{Introduction}
    \begin{quote}
        Mathematicians are people who would rather think than work.\newline
        \flushright{--David Grant}
    \end{quote}

    \bigskip

    First we shall start off with a definition:
    \begin{defn}
       \index{Abstraction}
       \underline{\emph{Abstraction}} is the process of producing a concept not associated with a specific instance.
    \end{defn}
    Why do we want this?
    Well it's simple, we would like to figure out how to build up concepts to permit us to predict, discover, and create new ideas.
    By grouping specific instances into concepts we can apply what we learn to a plethora of instances not just one.
    Also, we can use what other people have thought of and apply it to another instance without redoing the thinking.
    Besides, by abstracting we may discover new ideas which we could have never discovered otherwise.
    
    This brings us to our first main topic, mathematics.
    Mathematics is a philosophy used to bring together a vast number of topics into one topic so similarities between ideas may be exploited to discover the most elegant solution.
    For instance, ideas from calculus can be applied to geometry to analyze how shapes are, for lack of a better word, shaped.
    In regards to the quote above, math is a tool used to prevent "brute forcing" a solution.
    By "brute forcing" I mean trying every since possibility, basically guess and check with no direction at all.
    That would talk too much time.
    Remember if we can figure out a way to think of an problem in the most abstract we can more easily figure out the best way to solve it.

    So we start our endeavor first by ascertaining an idea of how to connect ideas.
    This topic is what we have come to call logic.

    \lipsum
