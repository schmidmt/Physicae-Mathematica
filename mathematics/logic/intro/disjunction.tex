\subsection{Disjunction (Or)}

Disjunction allows us to create an alternate to conjunction where we only require one constituent statement to be true to ensure the whole is also true.
This is represented by \emph{OR}.
Now, this is different from how we use \emph{OR} in informal language.
In informal language OR usually represents an \emph{exclusive OR (XOR)} (\ref{defn:XOR}) where only one condition can be true where as this OR represents any combination with a true statement is rendered true.

\begin{defn}
	\index{Disjunction}
  \index{Or|see{Disjunction}}
  \label{defn:Disjunction}
  \label{defn:OR}
  A logical \textbf{Disjunction (OR)}, $\lor$, combines two statements, say $A$ and $B$, in such a way that $A\lor B$ is true if either $A$ or $B$ is true (including the case where $A$ and $B$ are both true).
\end{defn}


The truth table for OR is given as follows:

\begin{minipage}{\linewidth}
  \centering
  \begin{tabular}{ccc}
    $A$ & $B$ & $A \lor B$ \\
    \toprule
    T & T & T \\
    T & F & T \\
    F & T & T \\
    F & F & F \\
  \end{tabular}
  \captionof{table}{Truth Table of Disjunction}\label{tab:disjunction} 
\end{minipage}

