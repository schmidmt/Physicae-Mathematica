\subsection{Negation}

Since we have this abstract way of representing statements, we can start to build up a tool-set to work with them.
We begin with our most simple modifier, the negation of a statement.

\begin{defn}{Negation}
  The \textbf{negation} of a sentence $S$ is true when $S$ is false and false when $S$ is true.
  We represent this negation of $S$ by $\neg S$.
\end{defn}


We can easily see how this could alter a statement by using a \textbf{Truth Table}\index{Truth Table}.
To create a Truth Table, we create columns for the input statement's truth value and create columns for each alteration.
For each row, lexicographically permute T and F to produce all forms of output.
Now, assign the alteration's with their values depending on the input's truth values.
The following table shows how this would work for the negation of $S$.

\begin{minipage}{\linewidth}
  \centering
  \begin{tabular}{cc}
    $S$ & $\neg S$ \\
    \toprule
    T & F \\
    F & T \\
  \end{tabular}
  \captionof{table}{Truth Table of Negation}\label{tab:negation} 
\end{minipage}


