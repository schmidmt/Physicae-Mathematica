\subsection{Conditional}


\begin{defn}{Conditional}
  A \textbf{conditional}, $ p \rightarrow q$ (IF $p$ THEN $q$), is a connective which is only false if and only if $p$ is true and $q$ is false.
  In connectives used so far, $p \rightarrow q$ is identical to $\neg (p \land \neg q)$.

  \begin{comment}
  \begin{itemize}
    \item[Modus Ponens] If $p$ is True, then the truth hinges on $q$.
    \item[Transitive] If $p \rightarrow q$ and $q \rightarrow r$, then $p \rightarrow r$ must also be true.
    \item[Contraposition] If $p \rightarrow q$ is True, then so must $\neg q \rightarrow p$.
    \item[Reductio]
  \end{itemize}
  \end{comment}

\end{defn}

The truth table for a conditional is as follows:

\begin{minipage}{\linewidth}
  \centering
  \begin{tabular}{ccc}
    $p$ & $q$ & $p \rightarrow q$ \\
    \hline
    T & T & T \\
    T & F & F \\
    F & T & T \\
    F & F & T \\
  \end{tabular}
  \captionof{table}{Truth Table for a Conditional} \label{tab:conditonal} 
\end{minipage}

