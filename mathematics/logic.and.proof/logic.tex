\section{Logic}
Since mathematics is entirely a human construction, proof is the only way to verify if an idea is true.
To prove an idea true, it must be shown to be logically (non-interpretively) true. To do this we first develop a statement which contains the idea, thereby bringing us to our first definition.

\begin{defn}
	\index{Statement}
	A \underline{\emph{statement}} is a phrase which is either true or false. A statement can be represented by a letter for simplicity and compactness.
\end{defn}

\begin{ex}
	The phrase P="The sky is blue" is a true statement.\\
	The phrase Q="The sky is brown" is a false statement.
\end{ex}

It is also useful to define the negation of a statement as follows:
\begin{defn}
	\index{Negation $\neg$}
	\nomenclature{$ \land $}{Logical And. True if all elements are true and false otherwise.}
	The \underline{\emph{Negation}} ($\neg$) of a statement is true when the statement is false and false if the statement is true.
	The statement $\neg P$ is pronounced "Not P".
\end{defn}

\begin{ex} Using the examples above:\\
	$\neg P=$"The sky is not blue", which is a false statement\\
	$\neg Q=$"The sky is not brown", which is a true statement.
\end{ex}

\subsection{And and Or}
This construction allows us to create simple statements, however it is often necessary to built more complicated statements to reflect more complex ideas.
First we can consider an operator that requires both statements to be true and will be false under any other circumstance.

\begin{defn}
	\index{And, Logical $\land$}
	\nomenclature{$ \lor $}{Logical Or. True if any elements is true and false otherwise.}
	A \underline{\emph{conjunction operator}}, or "And" ($\land$), connects logical statements and is true if all statements are true and false if any are false.
\end{defn}

The truth table for the conjunction operator is as follows:
\begin{center}
	\begin{tabular}{cc|c}
		$P$ & $Q$ & $P \land Q$ \\
		\hline
		T & T & T \\
		T & F & F \\
		F & T & F \\
		F & F & F \\
	\end{tabular}
\end{center}

So what happens if we take the negation of the statement $P \land Q$?
Well, the statement says, "Not both P and Q", which can be interpreted as "neither P nor Q", which again can be rewritten as "Not P or Not Q".
Using this idea or an \emph{or} connective we can define the disjunction operator as follows:

\begin{defn}
	\index{Or, Logical $\lor$}
	A \underline{\emph{disjunction operator}}, or "Or" ($\lor$), connects logical statements and is true if any statements are true and false if all are false.
\end{defn}

So lets look at the truth table for the disjunction operator with the conjunction operator.

\begin{center}
	\begin{tabular}{cc|ccc}
		$P$ & $Q$ & $P \land Q$ & $P \lor Q$ & $\neg P \lor \neg Q$ \\
		\hline
		T & T & T & T & F \\
		T & F & F & T & T \\
		F & T & F & T & T \\
		F & F & F & F & T \\
	\end{tabular}
\end{center}

From this we can see that $\neg (P \land Q)$ is equivalent to $\neg P \lor \neg Q$.
We can also look at the negation of the statement $P \lor Q$ by looking at it as the sentence, "Not either P or Q", and the only time that will work is when both P and Q are false, so the sentence becomes, "Not P and Not Q".
This can be rewritten as $\neg P \land \neg Q$.
This brings us to our first theorem, originally stated as laws by De Morgan.

\begin{thm}
	\index{De Morgani's Laws (Logic Version)}
	\textbf{(De Morgan's Laws)} \\
	\label{Thm:DeMorganLogic}
	\begin{equation}
		\nonumber
		\neg ( P \land Q ) \iff \neg P \lor \neg Q
	\end{equation}
	\begin{equation}
		\nonumber
		\neg ( P \lor Q ) \iff \neg P \land \neg Q
	\end{equation}
\end{thm}
Here the symbol $\iff$ means logically equivalent, we will generalize its meaning to "if and only if" operator later in the chapter.

\begin{proof}
	To prove theorem \ref{Thm:DeMorganLogic} we can construct the truth tables for the statements involved relatively simply.
	\begin{table*}[ht]
	\begin{center}
	\begin{tabular}{cc|cc|cccccc}
			$P$ & $Q$ & $\neg P$ & $\neg Q$ & $P \land Q$ & $\neg (P \land Q)$ & $P \lor Q$ & $\neg (P \lor Q)$ & $\neg P \lor \neg Q$ & $\neg P \land \neg Q$ \\
			\hline
			T & T & F & F & T & F & T & F & F & F \\
			T & F & F & T & F & T & T & F & T & F \\
			F & T & T & F & F & T & T & F & T & F \\
			F & F & T & T & F & T & F & T & T & T \\
	\end{tabular}
	\end{center}
	\end{table*}
	Since there are only four possible inputs for these operators, all possibilities have been shown and the theorem is proved.
\end{proof}

\begin{thm}
	\textbf{Properties of Logical Operators}\\
	\label{thm:proplogical}
	\textbf{Distributivity}: $(A \land B) \lor C \iff (A \lor C) \land (B \lor C)$ and $(A \lor B) \land C \iff (A \land C) \lor (B \land C)$
	
\end{thm}

\subsection{Implication}
We are ultimately concerned with using logic to show how two idea are connected.
Usually this connectedness is expressed in the form of a conditionals of the form, "If \emph{something} then \emph{something else}."
In terms of logic when a true statement causes another to also be true an implication is formed.
We can now define the implication as follows:

\begin{defn}
	\index{Implication, $ \Rightarrow $}
	An \emph{\underline{implication}} ($A \Rightarrow B$) between means that if A is true, so must B.
\end{defn}
With this definition, if A is false then B is unrestricted and can be anything.
So if A is false, the statement will always be true since a false statement implies anything.
We also note that if A is true but B is false, that would suggest that A doesn't imply B, so the statement is false.
Using this we can construct a truth table for an implication.

\begin{center}
	\begin{tabular}{cc|c}
		$P$ & $Q$ & $P \Rightarrow Q$ \\
		\hline
		T & T & T \\
		T & F & F \\
		F & T & T \\
		F & F & T \\
	\end{tabular}
\end{center}

So if $A \Rightarrow B$ is true, then a false B would imply a false A.
This brings us to the definition of the contrapositive.

\begin{defn}
	\index{Contrapositive}
	Given $A \Rightarrow B$, $\neg B \Rightarrow \neg A$ is called the contrapositive of $A \Rightarrow B$.
\end{defn}



