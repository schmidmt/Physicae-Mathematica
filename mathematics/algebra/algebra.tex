\chapter{Algebra}

\section{Groups}

\subsection{Introduction}

%TODO Write intro with lead-in and discussion of binary structures.

\begin{defn}
\index{Binary Structure}
\label{defn:binary_structure}
  A \textbf{binary structure} is a group $G$ endowed with a binary operation $\cdot$ represented by $(G, \cdot)$.
\end{defn}

% TODO: Add Introduction by way of symmetry groups and such things.

This section will be focused on the notion of a group.
In a high level view, a group is any binary structure for which there will always be a solution to $ A\cdot B = A\cdot C $, where $A,B,C \in G$.
It may not be immediately obvious how this could fail but lets imagine some scenarios to see where things can go wrong and determine what we require to be true for groups.

First, suppose $A\cdot B = D$ where $D \not\in G$. This immediately causes an issue as how can we know anything about how $D$ if it lies outside our set?
So, to fix this we stipulate that for all pairs of objects inside $G$, their ``product'' must also be inside $G$. 
This forms the first axiom of the group axioms, that of \emph{closure}.

Second, we require that the order of evaluation is insignificant i.e.\ the operations are associative.
Suppose if we did not require $A \cdot ( B \cdot C) = (A \cdot B) \cdot C$, we would have trouble later knowing how to apply the inverse to solve for a particular unknown.
This forms the second axiom of the group axioms, that of \emph{associativity}.
The problem here will become more evident by the end of the axioms.

Third and forth axioms deal with applying an inverse to solve the problem.
First we must have an inverse for each object in the set $G$ and we must also have an identity which will be the result of applying an object to it's inverse.
These two requirements create the third and forth group axioms: \emph{Identity} and \emph{Inevitability}.

We may define the group structure.
\begin{defn}
\index{Group}
\label{defn:group}
  A binary structure $(G, \cdot)$, usually denoted just $G$ for groups, is called a \textbf{group} if and only if $G$ satisfies the group axioms (\ref{axioms:group}).
\end{defn}

\begin{axiomset}
\label{axioms:group}
  \textbf{(Group Axioms)}
  For a binary structure (\ref{defn:binary_structure}) to be considered a \textbf{group}, it must satisfy the following axioms.
  \begin{axiom}
     \textbf{Closure:}
     \begin{equation}
       \forall x,y \in G \quad x\cdot y \in G
     \end{equation}
  \end{axiom}
  \begin{axiom}
     \textbf{Associativity:}
     \begin{equation}
       \forall x,y,z \in G \quad (x\cdot y) \cdot z = z \cdot (y \cdot z)
     \end{equation}
  \end{axiom}
  \begin{axiom}
     \textbf{Identity:}
     \begin{equation}
       \exists e \in G \quad \forall x \in G \quad e\cdot x = x \cdot e = x
     \end{equation}
  \end{axiom}
  \begin{axiom}
     \textbf{Invertability:}
     \begin{equation}
       \forall x \in G \quad \exists x^{-1} \in G \quad x \cdot x^{-1} = x^{-1} \cdot x = e
     \end{equation}
  \end{axiom}
\end{axiomset}

\begin{ex}
  The simplest group is that of one element (yes, you can have a mathematical group of one element).
  For this example, suppose $e \in G$ is the set in question.
  If $e$ is the identity, then $e \cdot e = e$ which is in the group.
  Additionally, $e^{-1} = e$ so we have an inverse.
  This all means $G$ is a group.
\end{ex}

\begin{ex}
  An example of a group you probably have experience with is the integers ($\mathbb{Z}$) endowed with addition ($+$).
  The set is closed and addition is associative, so axioms I and II are satisfied.
  The additive inverse is $0$ since $\forall x \in \mathbb{Z} \; x + 0 = 0 + x = x $ and each number, $x$, has an inverse, $-x$, such that $\forall x in \mathbb{Z} \; x + (-x) = (-x) + x = 0$.
  These satisfy axioms III and IV\@.
  Therefore $\mathbb{Z}$ with addition forms a group.
\end{ex}

\begin{ex}
\index{Klein four-group}
\index{Vierergruppe see Klein four-group}
\label{ex:vierergruppe}
  \textbf{Klein four-group (German: Vierergruppe)}
  The Klein four-group, usually denoted $V$, is the smallest non-cyclic group (we'll get to that definition in a moment).
  The \textbf{Cayley Table} for $V$ is given below:

  \begin{center}
  \begin{tabular}{c|cccc}
    $\star$ & 1 & a & b & ab \\
    \toprule
    1       & 1  & a  & b  & ab \\
    a       & a  & 1  & ab & b \\
    b       & b  & ab & 1  & a \\
    ab      & ab & b  & a  & 1 
  \end{tabular}
  \end{center}

  This group is interesting since it's form is similar\footnote{Actually, they're isomorphic.} to many other groups.
  For instance, $V$ is similar to the symmetry group for rhombi, not square, where the elements are the identity, reflection across the horizontal, reflection across the vertical, and $180^\circ$ rotation.
\end{ex}

\begin{comment}
This is the Cayley Digraph for the Klein four-group.
\begin{tikzpicture}[>=latex',scale=1]
    \begin{dot2tex}[dot,tikz,codeonly,styleonly,options=-s -tmath]
        digraph G{
            node [shape="circle"];
            1 -> 1 [label="1"];
            1 -> a [label="a"];
            1 -> b [label="b"];
            1 -> ab [label="ab"];

            a -> a [label="1"];
            a -> 1 [label="a"];
            a -> ab [label="b"];
            a -> b [label="ab"];

            b -> 1 [label="b"];
            b -> b [label="1"];
            b -> a [label="ab"];
            b -> ab [label="a"];

            ab -> a [label="b"];
            ab -> b [label="a"];
            ab -> ab [label="1"];
            ab -> 1 [label="ab"];
            {rank=same; 1;b}
            {rank=same; a;ab}
        }
    \end{dot2tex}.
\end{tikzpicture}
\end{comment}


\begin{ex}
  Here is an example where not all group axioms are satisfied:
  Suppose we take $(\mathbb{N}, +)$.
  We have closure, and associativity but we lack inverses.
  If we had an equation like $1+A=2+A$ it's obvious there is no solution in $\mathbb{N}$.
\end{ex}

Since we now know about the basic definition of a group, we can start to explore different types of groups.
For instance, if a group can be entirely generated by a single element repeatedly applied to itself we call the formed group a \textbf{cyclic group}.

\begin{defn}
\index{Cyclic Group}
\label{defn:cyclicGroup}
  A \textbf{cyclic group} is a group, $G$, such that $\exists a \in G \; \forall x \in G \; \exists n \in \mathbb{Z} \; a^n = x$ where $a^n = \overbrace{a \cdot a \cdots a}^{n\text{times}}$.
  If $a$ generates $G$ we say $a$ is a generator of $G$.
\end{defn}

\begin{ex}
  $\mathbb{Z}_n$ is the \textbf{Multiplicative Group of Integers Modulo $n$}.
  This set contains all integers from $0$ to $n-1$ where the operation is defined to be $a \cdot b = a \times b \; (mod \; n)$.
  For $\mathbb{Z}_4$, the multiplication table is as follows:

  \begin{center}
  \begin{tabular}{c|cccc}
      & 0 & 1 & 2 & 3 \\
    \toprule
    0 & 0 & 0 & 0 & 0 \\
    1 & 0 & 1 & 2 & 3 \\
    2 & 0 & 2 & 0 & 2 \\
    3 & 0 & 3 & 2 & 1
  \end{tabular}
  \end{center}

\end{ex}



\section{Rings}

\section{Fields}

