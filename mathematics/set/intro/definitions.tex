\subsection{Notation}
As with any subject, we must establish a common language for discussing the subject.
Sets are often discussed by referencing their content explicitly, as in the case above, or implicitly.
Explicit set definitions are usually only useful when working with small, simple sets.
For example we can denote the set of integers from 1 to 10 by 
$$
A = \{1,2,3,4,5,6,7,8,9,10\}
$$

This is good for small sets of elements or where the number elements are small.
However, as we now understand how to use logic to stipulate rules we can use a logical predicate to define sets.
For the set $A$ defined above we could have said, $A = \{\text{the set of integers from 1 to 10}\}$.
To simplify this we could have set define a formula $\phi = 1 \le x \le 10 \land x \in \mathbb{Z}$ and defined $A$ as all values of $x$ which make $\phi$ true.
The method being developed here is called \textbf{Set-Builder Notation} which is defined here:
\begin{defn}{Set-Builder Notation}
  \textbf{Set-Builder Notation} defines a set, $A$, using the following form:
  $$
  A = \{x|\phi(x)\}
  $$
  where $\phi(x)$ is a logical predicate and $x$ is a free variable.
  A logically equivalent definition for $A$ is: $x \in A \iff \phi(x)$.
\end{defn}

\begin{ex}
  Suppose we wanted to define a set such that all elements were square numbers.
  We first need to develop a predicate to establish membership.
  A possible predicate is the following: $\phi(x) = \exists n \in \mathbb{Z} (n*n = x)$.
  In set builder notation this would be
  $$
    \{ x | \exists n \in \mathbb{Z} (n*n = x) \}
  $$
\end{ex}

The two reasons set builder notation is valuable is it makes explicit the connection between set theory and logic and it allows us to easily define large and infinite sets.
A set like the one in the example above cannot be explicitly defined as it would require an infinitely long book, of which you could never finish, so why do that.


\subsection{Some Definitions}
Sets, like logical sentences, can interact with each other and be altered to create new sets.
A few basic concepts will be conveyed here to give a foundation for discussion later.


Since any collection can be subdivided into other collections it is useful to make the notion of a \textbf{subset} strict.
To exemplify this, imagine two sets $A$ and $B$.
If each element of $A$ is in $B$ then $A$ is contained within $B$ or, in other words, $A$ is a subset of $B$.
To make this formal, the definition of a subset is as follows:
\begin{defn}{Subset}
  Suppose $A$ and $B$ are two sets.
  For each $a$ if $a$ is in $A$ and it is also in $B$ then $A$ is a \textbf{subset} of $B$.
  Symbolically this case is represented as $A \subset B$.
  Formally, this is represented by:
  $$
    \forall a \left\{ a \in A \implies a \in B \right\}  \Leftrightarrow A \subset B.
  $$
\end{defn}

Having a concept of a subset, we can now define the power set of a set.
\begin{defn}{Power Set}
  Let $A$ be a set.
  The \textbf{power set} of $A$, or $\mathcal{P}(A)$, is the set of all subsets of $A$.
  Formally, this is defined by:
  $$
    \forall x \left\{ x \in \mathcal{P}(A) \iff x \subset A \right\}.
  $$
  or in set-builder notation,
  $$
    \mathcal{P}(A) = \left\{x | x \subset A \right\}
  $$
\end{defn}

If two sets are mutually subsets of each other we say the sets are equivalent.
\begin{defn}{Set Equality}
  Suppose $A$ and $B$ are two sets.
  If $A \subset B$ and $B \subset A$ then $A$ and $B$ are \textbf{equal}, or $A = B$.
\end{defn}
This might at first seems a bit strange as the set $\left\{ a, a, b\right\}$ is the same as the set $\left\{ a, b \right\}$.
In this way, a set with repeated elements or elements in different orders are considered the same.
Later we will discuss different types of objects which do retain these qualities.

We may combine two sets in two distinct ways: we could group everything together, or we could select just the common elements.
The first way, grouping everything together is called the union of two sets as more precisely defined here:
\begin{defn}{Union}
\nomenclature{Union}{The union of two sets is the set of elements in both sets}
  Suppose $A$ and $B$ are two sets.
  The \textbf{union} of $A$ and $B$ is the set of all elements in at least one of the two sets.
  This is symbolically represented as $A \union B$ and is defined by the following sentence:
  $$
  \forall a \left\{ a \in A \union B \iff (a \in A \lor a \in B) \right\}.
  $$
  Using set builder notation, it can be defined as follows:
  $$
    A \union B =  \left\{ a | a \in A \lor a \in B \right\}.
  $$
\end{defn}

For the second way to combine two sets we only collect those elements which are in both sets.
We call this the \textbf{intersection} of two sets.
Formally, this is defined by:
\begin{defn}{Intersection}
\nomenclature{Intersection}{The intersection of two sets is the set of common elements}
  Suppose $A$ and $B$ are two sets.
  The \textbf{intersection} of $A$ and $B$ is the set of all elements at both sets.
  This is symbolically represented as $A \intersection B$ and is defined by the following sentence:
  $$
  \forall a \left\{ a \in A \intersection B \iff (a \in A \land a \in B) \right\}.
  $$
  Using set builder notation, it can be defined as follows:
  $$
    A \intersection B =  \left\{ a | a \in A \land a \in B \right\}.
  $$
\end{defn}


The next definitions focus on connecting elements within and between sets.
Often it is helpful to define a set objects between two sets.
If we choose an element $a$ from set $A$ and $b$ from set $B$ we can place them in a object we call an ordered-pair.
\begin{defn}{Ordered-Pair}
    An \textbf{ordered-pair} is a pair of objects $a$ and $b$ written $(a,b)$ where the order creates distinction.
    Ordering requires $(a,b) \ne (b,a)$ if and only if $a \ne b$.
\end{defn}

A set of ordered pairs from two sets $A$ and $B$ is called a Cartesian Product.
\begin{defn}{Cartesian Product}
    The \textbf{Cartesian Product} of two sets, $A$ with $B$, written $A \times B$ is defined to by:
    $$
        A \times B = \left\{ (a, b) | a \in A \land b \in B \right\}
    $$
\end{defn}

Next, we consider relationships between two elements of a set.
A relationship is a subset of the Cartesian Product between two sets and can be used to establish a connection between elements.

\begin{defn}{Relation}
    A \textbf{relation}, $R$, between two sets, $A$ and $B$, is a subset of $A \times B$.
    $(a, b) \in R$ is usually written $a$ is $R$-related to $b$.
    Symbolically, $aRb = R(a,b)$ and is true if an only if $(a,b) \in R$.
    In the first case we conceptualize $R$ as a binary operator which yields a boolean value.
    In the second case we can think of $R(a,b)$ as a boolean function of two variables.
\end{defn}
If $A = B$, then the relation is called an \textbf{endorelation}\index{Endorelation}.

Endorelations\index{Endorelation}\index{Relation!Endo-} can have the following attributes:

\begin{tabular}{ll}
    \centering
    Attribute & Condition \\
    \toprule
    Reflexive       \index{Relation!Reflexive}       & $\forall x \in X (xRx)$ \\
    Irreflexive     \index{Relation!Irreflexive}     & $\forall x \in X \lnot (xRx)$ \\
    Symmetric       \index{Relation!Symmetric}       & $\forall x,y \in X (xRy \iff yRx) $ \\
    Antisymmetric   \index{Relation!Antisymmetric}   & $\forall x,y \in X ((xRy \land yRx) \iff x=y ) $ \\
    Asymetric       \index{Relation!Asymetric}       & $\forall x,y \in X (xRy \iff \lnot yRx) $ \\
    Transitive      \index{Relation!Transitive}      & $\forall x,y,z \in X ((xRy \land yRz) \implies xRz)$ \\
    Total           \index{Relation!Total}           & $\forall x,y \in X (xRy \lor yRx) $ \\
    Trichotomous    \index{Relation!Trichotomous}    & $\forall x,y \in X (xRy \lor yRx \lor x=y) $ \\
    Left Euclidean  \index{Relation!Left Euclidean}  & $\forall x,y,z \in X ((yRx \land zRx) \implies yRz)$ \\
    Right Euclidean \index{Relation!Right Euclidean} & $\forall x,y,z \in X ((xRy \land xRz) \implies yRz)$ \\
    Euclidean       \index{Relation!Euclidean}       & Relation is both Left and Right Euclidean.
\end{tabular}

